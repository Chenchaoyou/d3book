\documentclass[]{book}
\usepackage{lmodern}
\usepackage{amssymb,amsmath}
\usepackage{ifxetex,ifluatex}
\usepackage{fixltx2e} % provides \textsubscript
\ifnum 0\ifxetex 1\fi\ifluatex 1\fi=0 % if pdftex
  \usepackage[T1]{fontenc}
  \usepackage[utf8]{inputenc}
\else % if luatex or xelatex
  \ifxetex
    \usepackage{mathspec}
  \else
    \usepackage{fontspec}
  \fi
  \defaultfontfeatures{Ligatures=TeX,Scale=MatchLowercase}
\fi
% use upquote if available, for straight quotes in verbatim environments
\IfFileExists{upquote.sty}{\usepackage{upquote}}{}
% use microtype if available
\IfFileExists{microtype.sty}{%
\usepackage{microtype}
\UseMicrotypeSet[protrusion]{basicmath} % disable protrusion for tt fonts
}{}
\usepackage[margin=1in]{geometry}
\usepackage{hyperref}
\hypersetup{unicode=true,
            pdftitle={D3 for R Users},
            pdfauthor={Joyce Robbins},
            pdfborder={0 0 0},
            breaklinks=true}
\urlstyle{same}  % don't use monospace font for urls
\usepackage{color}
\usepackage{fancyvrb}
\newcommand{\VerbBar}{|}
\newcommand{\VERB}{\Verb[commandchars=\\\{\}]}
\DefineVerbatimEnvironment{Highlighting}{Verbatim}{commandchars=\\\{\}}
% Add ',fontsize=\small' for more characters per line
\usepackage{framed}
\definecolor{shadecolor}{RGB}{248,248,248}
\newenvironment{Shaded}{\begin{snugshade}}{\end{snugshade}}
\newcommand{\AlertTok}[1]{\textcolor[rgb]{0.94,0.16,0.16}{#1}}
\newcommand{\AnnotationTok}[1]{\textcolor[rgb]{0.56,0.35,0.01}{\textbf{\textit{#1}}}}
\newcommand{\AttributeTok}[1]{\textcolor[rgb]{0.77,0.63,0.00}{#1}}
\newcommand{\BaseNTok}[1]{\textcolor[rgb]{0.00,0.00,0.81}{#1}}
\newcommand{\BuiltInTok}[1]{#1}
\newcommand{\CharTok}[1]{\textcolor[rgb]{0.31,0.60,0.02}{#1}}
\newcommand{\CommentTok}[1]{\textcolor[rgb]{0.56,0.35,0.01}{\textit{#1}}}
\newcommand{\CommentVarTok}[1]{\textcolor[rgb]{0.56,0.35,0.01}{\textbf{\textit{#1}}}}
\newcommand{\ConstantTok}[1]{\textcolor[rgb]{0.00,0.00,0.00}{#1}}
\newcommand{\ControlFlowTok}[1]{\textcolor[rgb]{0.13,0.29,0.53}{\textbf{#1}}}
\newcommand{\DataTypeTok}[1]{\textcolor[rgb]{0.13,0.29,0.53}{#1}}
\newcommand{\DecValTok}[1]{\textcolor[rgb]{0.00,0.00,0.81}{#1}}
\newcommand{\DocumentationTok}[1]{\textcolor[rgb]{0.56,0.35,0.01}{\textbf{\textit{#1}}}}
\newcommand{\ErrorTok}[1]{\textcolor[rgb]{0.64,0.00,0.00}{\textbf{#1}}}
\newcommand{\ExtensionTok}[1]{#1}
\newcommand{\FloatTok}[1]{\textcolor[rgb]{0.00,0.00,0.81}{#1}}
\newcommand{\FunctionTok}[1]{\textcolor[rgb]{0.00,0.00,0.00}{#1}}
\newcommand{\ImportTok}[1]{#1}
\newcommand{\InformationTok}[1]{\textcolor[rgb]{0.56,0.35,0.01}{\textbf{\textit{#1}}}}
\newcommand{\KeywordTok}[1]{\textcolor[rgb]{0.13,0.29,0.53}{\textbf{#1}}}
\newcommand{\NormalTok}[1]{#1}
\newcommand{\OperatorTok}[1]{\textcolor[rgb]{0.81,0.36,0.00}{\textbf{#1}}}
\newcommand{\OtherTok}[1]{\textcolor[rgb]{0.56,0.35,0.01}{#1}}
\newcommand{\PreprocessorTok}[1]{\textcolor[rgb]{0.56,0.35,0.01}{\textit{#1}}}
\newcommand{\RegionMarkerTok}[1]{#1}
\newcommand{\SpecialCharTok}[1]{\textcolor[rgb]{0.00,0.00,0.00}{#1}}
\newcommand{\SpecialStringTok}[1]{\textcolor[rgb]{0.31,0.60,0.02}{#1}}
\newcommand{\StringTok}[1]{\textcolor[rgb]{0.31,0.60,0.02}{#1}}
\newcommand{\VariableTok}[1]{\textcolor[rgb]{0.00,0.00,0.00}{#1}}
\newcommand{\VerbatimStringTok}[1]{\textcolor[rgb]{0.31,0.60,0.02}{#1}}
\newcommand{\WarningTok}[1]{\textcolor[rgb]{0.56,0.35,0.01}{\textbf{\textit{#1}}}}
\usepackage{graphicx,grffile}
\makeatletter
\def\maxwidth{\ifdim\Gin@nat@width>\linewidth\linewidth\else\Gin@nat@width\fi}
\def\maxheight{\ifdim\Gin@nat@height>\textheight\textheight\else\Gin@nat@height\fi}
\makeatother
% Scale images if necessary, so that they will not overflow the page
% margins by default, and it is still possible to overwrite the defaults
% using explicit options in \includegraphics[width, height, ...]{}
\setkeys{Gin}{width=\maxwidth,height=\maxheight,keepaspectratio}
\IfFileExists{parskip.sty}{%
\usepackage{parskip}
}{% else
\setlength{\parindent}{0pt}
\setlength{\parskip}{6pt plus 2pt minus 1pt}
}
\setlength{\emergencystretch}{3em}  % prevent overfull lines
\providecommand{\tightlist}{%
  \setlength{\itemsep}{0pt}\setlength{\parskip}{0pt}}
\setcounter{secnumdepth}{0}
% Redefines (sub)paragraphs to behave more like sections
\ifx\paragraph\undefined\else
\let\oldparagraph\paragraph
\renewcommand{\paragraph}[1]{\oldparagraph{#1}\mbox{}}
\fi
\ifx\subparagraph\undefined\else
\let\oldsubparagraph\subparagraph
\renewcommand{\subparagraph}[1]{\oldsubparagraph{#1}\mbox{}}
\fi

%%% Use protect on footnotes to avoid problems with footnotes in titles
\let\rmarkdownfootnote\footnote%
\def\footnote{\protect\rmarkdownfootnote}

%%% Change title format to be more compact
\usepackage{titling}

% Create subtitle command for use in maketitle
\providecommand{\subtitle}[1]{
  \posttitle{
    \begin{center}\large#1\end{center}
    }
}

\setlength{\droptitle}{-2em}

  \title{D3 for R Users}
    \pretitle{\vspace{\droptitle}\centering\huge}
  \posttitle{\par}
    \author{Joyce Robbins}
    \preauthor{\centering\large\emph}
  \postauthor{\par}
      \predate{\centering\large\emph}
  \postdate{\par}
    \date{2019-08-06}


\begin{document}
\maketitle

\hypertarget{jump-in-the-deep-end}{%
\chapter{Jump in the deep end}\label{jump-in-the-deep-end}}

Let's skip the explanations and start coding in D3 right now. Why? So
you can see the benefits and know what you're working toward when you
get stuck in the weeds. Then we'll go back and start learning step by
step.

\hypertarget{what-you-need}{%
\subsection{What you need}\label{what-you-need}}

\href{https://www.google.com/chrome/}{Chrome browser}

\hypertarget{elements}{%
\subsection{Elements}\label{elements}}

\begin{enumerate}
\def\labelenumi{\arabic{enumi}.}
\tightlist
\item
  Open a \emph{downloaded version} of \texttt{shapes.html}.
\end{enumerate}

You can obtain a downloaded copy of the file by:

\begin{itemize}
\tightlist
\item
  clicking
  \href{https://raw.githubusercontent.com/jtr13/d3book/master/code/shapes.html}{here}
  and then File, Save Page As\ldots{}
\end{itemize}

\textbf{or}

\begin{itemize}
\tightlist
\item
  downloading a ZIP of the whole repo by clicking
  \href{https://github.com/jtr13/d3book/archive/master.zip}{here}
\end{itemize}

\textbf{or}

\begin{itemize}
\tightlist
\item
  forking and cloning the \href{https://github.com/jtr13/d3book}{repo}
\end{itemize}

\begin{enumerate}
\def\labelenumi{\arabic{enumi}.}
\setcounter{enumi}{1}
\item
  Click \emph{View, Developer, Developer Tools,} then the
  \texttt{Elements} tab.
\item
  Hover the mouse over various elements in the
  \texttt{\textless{}body\textgreater{}\ ...\ \textless{}/body\textgreater{}}
  section.
\item
  Click the Console tab. Type
  \texttt{d3.select("circle").attr("cx",\ "200");} at the prompt
  (\texttt{\textgreater{}}), press enter and see what happens.
\item
  Now try some of the following and/or experiment on your own with
  changing attributes of the circle:

\begin{Shaded}
\begin{Highlighting}[]
\VariableTok{d3}\NormalTok{.}\AttributeTok{select}\NormalTok{(}\StringTok{"circle"}\NormalTok{).}\AttributeTok{attr}\NormalTok{(}\StringTok{"cx"}\OperatorTok{,} \StringTok{"300"}\NormalTok{)}\OperatorTok{;}

\VariableTok{d3}\NormalTok{.}\AttributeTok{select}\NormalTok{(}\StringTok{"circle"}\NormalTok{).}\AttributeTok{attr}\NormalTok{(}\StringTok{"cx"}\OperatorTok{,} \StringTok{"400"}\NormalTok{)}\OperatorTok{;}

\VariableTok{d3}\NormalTok{.}\AttributeTok{select}\NormalTok{(}\StringTok{"circle"}\NormalTok{).}\AttributeTok{attr}\NormalTok{(}\StringTok{"cx"}\OperatorTok{,} \StringTok{"500"}\NormalTok{)}\OperatorTok{;}

\VariableTok{d3}\NormalTok{.}\AttributeTok{select}\NormalTok{(}\StringTok{"circle"}\NormalTok{).}\AttributeTok{attr}\NormalTok{(}\StringTok{"cx"}\OperatorTok{,} \StringTok{"600"}\NormalTok{)}\OperatorTok{;}

\VariableTok{d3}\NormalTok{.}\AttributeTok{select}\NormalTok{(}\StringTok{"circle"}\NormalTok{).}\AttributeTok{attr}\NormalTok{(}\StringTok{"cx"}\OperatorTok{,} \StringTok{"100"}\NormalTok{)}\OperatorTok{;}

\VariableTok{d3}\NormalTok{.}\AttributeTok{select}\NormalTok{(}\StringTok{"circle"}\NormalTok{).}\AttributeTok{attr}\NormalTok{(}\StringTok{"r"}\OperatorTok{,} \StringTok{"30"}\NormalTok{)}\OperatorTok{;}

\VariableTok{d3}\NormalTok{.}\AttributeTok{select}\NormalTok{(}\StringTok{"circle"}\NormalTok{).}\AttributeTok{attr}\NormalTok{(}\StringTok{"r"}\OperatorTok{,} \StringTok{"130"}\NormalTok{)}\OperatorTok{;}

\VariableTok{d3}\NormalTok{.}\AttributeTok{select}\NormalTok{(}\StringTok{"circle"}\NormalTok{).}\AttributeTok{attr}\NormalTok{(}\StringTok{"r"}\OperatorTok{,} \StringTok{"3"}\NormalTok{)}\OperatorTok{;}

\VariableTok{d3}\NormalTok{.}\AttributeTok{select}\NormalTok{(}\StringTok{"circle"}\NormalTok{).}\AttributeTok{attr}\NormalTok{(}\StringTok{"r"}\OperatorTok{,} \StringTok{"30"}\NormalTok{)}\OperatorTok{;}

\VariableTok{d3}\NormalTok{.}\AttributeTok{select}\NormalTok{(}\StringTok{"circle"}\NormalTok{).}\AttributeTok{attr}\NormalTok{(}\StringTok{"fill"}\OperatorTok{,} \StringTok{"red"}\NormalTok{)}\OperatorTok{;}

\VariableTok{d3}\NormalTok{.}\AttributeTok{select}\NormalTok{(}\StringTok{"circle"}\NormalTok{).}\AttributeTok{attr}\NormalTok{(}\StringTok{"fill"}\OperatorTok{,} \StringTok{"aliceblue"}\NormalTok{)}\OperatorTok{;}

\VariableTok{d3}\NormalTok{.}\AttributeTok{select}\NormalTok{(}\StringTok{"circle"}\NormalTok{).}\AttributeTok{attr}\NormalTok{(}\StringTok{"fill"}\OperatorTok{,} \StringTok{"lightseagreen"}\NormalTok{)}\OperatorTok{;}
\end{Highlighting}
\end{Shaded}
\item
  Refresh the page. What happened?
\item
  Go to Elements. Look at the value of the \texttt{y1} attribute of the
  SVG \texttt{\textless{}line\textgreater{}} element. Go back to the
  Console and enter the following:

\begin{Shaded}
\begin{Highlighting}[]
\VariableTok{d3}\NormalTok{.}\AttributeTok{select}\NormalTok{(}\StringTok{"line"}\NormalTok{).}\AttributeTok{attr}\NormalTok{(}\StringTok{"y1"}\OperatorTok{,} \StringTok{"10"}\NormalTok{)}\OperatorTok{;}
\end{Highlighting}
\end{Shaded}
\item
  Go back to Elements and observe. What happened?
\item
  Stay in Elements and refresh the page. What happened to \texttt{y1}?
\item
  Now back to the Console to make style changes to the HTML elements:

\begin{Shaded}
\begin{Highlighting}[]
\VariableTok{d3}\NormalTok{.}\AttributeTok{select}\NormalTok{(}\StringTok{"h1"}\NormalTok{).}\AttributeTok{style}\NormalTok{(}\StringTok{"color"}\OperatorTok{,} \StringTok{"purple"}\NormalTok{)}\OperatorTok{;}

\VariableTok{d3}\NormalTok{.}\AttributeTok{select}\NormalTok{(}\StringTok{"h2"}\NormalTok{).}\AttributeTok{style}\NormalTok{(}\StringTok{"font-size"}\OperatorTok{,} \StringTok{"50px"}\NormalTok{)}\OperatorTok{;}

\VariableTok{d3}\NormalTok{.}\AttributeTok{select}\NormalTok{(}\StringTok{"h2"}\NormalTok{).}\AttributeTok{style}\NormalTok{(}\StringTok{"font-family"}\OperatorTok{,} \StringTok{"Impact"}\NormalTok{)}\OperatorTok{;}
\end{Highlighting}
\end{Shaded}
\end{enumerate}

\hypertarget{part-b-transitions-ch.-9}{%
\subsection{Part B: Transitions (Ch.
9)}\label{part-b-transitions-ch.-9}}

\begin{enumerate}
\def\labelenumi{\arabic{enumi}.}
\item
  Try these:

\begin{Shaded}
\begin{Highlighting}[]
\VariableTok{d3}\NormalTok{.}\AttributeTok{select}\NormalTok{(}\StringTok{"circle"}\NormalTok{).}\AttributeTok{transition}\NormalTok{().}\AttributeTok{duration}\NormalTok{(}\DecValTok{2000}\NormalTok{).}\AttributeTok{attr}\NormalTok{(}\StringTok{"cx"}\OperatorTok{,} \StringTok{"400"}\NormalTok{)}\OperatorTok{;}

\VariableTok{d3}\NormalTok{.}\AttributeTok{select}\NormalTok{(}\StringTok{"ellipse"}\NormalTok{).}\AttributeTok{transition}\NormalTok{().}\AttributeTok{duration}\NormalTok{(}\DecValTok{2000}\NormalTok{).}\AttributeTok{attr}\NormalTok{(}\StringTok{"transform"}\OperatorTok{,} \StringTok{"translate (400, 400)"}\NormalTok{)}\OperatorTok{;}

\VariableTok{d3}\NormalTok{.}\AttributeTok{select}\NormalTok{(}\StringTok{"line"}\NormalTok{).}\AttributeTok{transition}\NormalTok{().}\AttributeTok{duration}\NormalTok{(}\DecValTok{2000}\NormalTok{).}\AttributeTok{attr}\NormalTok{(}\StringTok{"x1"}\OperatorTok{,} \StringTok{"400"}\NormalTok{)}\OperatorTok{;}

\VariableTok{d3}\NormalTok{.}\AttributeTok{select}\NormalTok{(}\StringTok{"line"}\NormalTok{).}\AttributeTok{transition}\NormalTok{().}\AttributeTok{duration}\NormalTok{(}\DecValTok{2000}\NormalTok{).}\AttributeTok{attr}\NormalTok{(}\StringTok{"y1"}\OperatorTok{,} \StringTok{"250"}\NormalTok{)}\OperatorTok{;}

\VariableTok{d3}\NormalTok{.}\AttributeTok{select}\NormalTok{(}\StringTok{"p"}\NormalTok{).}\AttributeTok{transition}\NormalTok{().}\AttributeTok{duration}\NormalTok{(}\DecValTok{2000}\NormalTok{).}\AttributeTok{style}\NormalTok{(}\StringTok{"font-size"}\OperatorTok{,} \StringTok{"72px"}\NormalTok{)}\OperatorTok{;}
\end{Highlighting}
\end{Shaded}
\item
  Experiment with more transitions.
\end{enumerate}

\hypertarget{part-c-interactivity-ch.-10}{%
\subsection{Part C: Interactivity (Ch.
10)}\label{part-c-interactivity-ch.-10}}

\begin{enumerate}
\def\labelenumi{\arabic{enumi}.}
\item
  Set up a function to turn the fill color to yellow:

\begin{Shaded}
\begin{Highlighting}[]
\KeywordTok{function} \AttributeTok{goyellow}\NormalTok{() }\OperatorTok{\{}\VariableTok{d3}\NormalTok{.}\AttributeTok{select}\NormalTok{(}\KeywordTok{this}\NormalTok{).}\AttributeTok{attr}\NormalTok{(}\StringTok{"fill"}\OperatorTok{,} \StringTok{"yellow"}\NormalTok{)}\OperatorTok{\};}
\end{Highlighting}
\end{Shaded}
\item
  Add an event listener to the circle that will be trigger a call to
  \texttt{goyellow()} on a mouseover:

\begin{Shaded}
\begin{Highlighting}[]
\VariableTok{d3}\NormalTok{.}\AttributeTok{select}\NormalTok{(}\StringTok{"circle"}\NormalTok{).}\AttributeTok{on}\NormalTok{(}\StringTok{"mouseover"}\OperatorTok{,}\NormalTok{ goyellow)}\OperatorTok{;}
\end{Highlighting}
\end{Shaded}
\item
  Test it out.
\item
  Add the same event listener to the ellipse. Test it out.
\item
  Create a function \texttt{goblue()} that changes the fill color to
  blue.
\item
  Add event listeners to the circle and ellipse that will trigger a call
  to \texttt{goblue()} on a \emph{mouseout}. Test out your code.
\item
  Try out a click event. (Note the use of an anonymous function.)

\begin{Shaded}
\begin{Highlighting}[]
\VariableTok{d3}\NormalTok{.}\AttributeTok{select}\NormalTok{(}\StringTok{"line"}\NormalTok{).}\AttributeTok{on}\NormalTok{(}\StringTok{"click"}\OperatorTok{,} \KeywordTok{function}\NormalTok{()}
  \OperatorTok{\{}\VariableTok{d3}\NormalTok{.}\AttributeTok{select}\NormalTok{(}\KeywordTok{this}\NormalTok{).}\AttributeTok{attr}\NormalTok{(}\StringTok{"stroke-width"}\OperatorTok{,} \StringTok{"10"}\NormalTok{)}\OperatorTok{;\}}\NormalTok{)}\OperatorTok{;}
\end{Highlighting}
\end{Shaded}
\item
  Try another click event. What's happening?

\begin{Shaded}
\begin{Highlighting}[]
\VariableTok{d3}\NormalTok{.}\AttributeTok{select}\NormalTok{(}\StringTok{"svg"}\NormalTok{).}\AttributeTok{on}\NormalTok{(}\StringTok{"click"}\OperatorTok{,} \KeywordTok{function}\NormalTok{()}
  \OperatorTok{\{}\VariableTok{d3}\NormalTok{.}\AttributeTok{select}\NormalTok{(}\StringTok{"text"}\NormalTok{).}\AttributeTok{text}\NormalTok{(}\VerbatimStringTok{`(}\SpecialCharTok{$\{}\VariableTok{d3}\NormalTok{.}\AttributeTok{mouse}\NormalTok{(}\KeywordTok{this}\NormalTok{)}\SpecialCharTok{\}}\VerbatimStringTok{)`}\NormalTok{)}\OperatorTok{\}}\NormalTok{)}\OperatorTok{;}
\end{Highlighting}
\end{Shaded}
\end{enumerate}

\hypertarget{introduction}{%
\chapter{Introduction}\label{introduction}}

Although it has plenty of new material and exercises, this book
frequently references Scott Murray's
\href{https://www.amazon.com/Interactive-Data-Visualization-Web-Introduction/dp/1491921285/}{\emph{Interactive
Data Visualization for the Web, 2nd edition}}, a required book for
GR5702. As such this resource is more a supplement to \emph{IDVW2}
rather than an alternative.

\emph{IDVW2} is the gold standard for learning D3. You may be wondering,
why do we need any extra material? Here's why:

\begin{itemize}
\item
  \emph{IDVW2} is written for graphics designers not data science
  students so the pain points are somewhat different. As the title
  states, my intended audience is R users, though you certainly don't
  need to know R to use this resource.
\item
  \emph{IDVW2} does not use certain ES6 conventions which make coding
  easier (and more like R 🤩).
\item
  \emph{IDVW2} does not include examples involved advanced statistics.
\item
  I prefer presenting some of the material in a different order than
  presented in \emph{IDVW2}.
\end{itemize}

None of this detracts from the fact that \emph{IDVW2} is absolutely
essential to learning D3. Although we will consider different examples,
you are encouraged to study
\href{https://github.com/alignedleft/d3-book/releases}{Murray's code
examples} in addition to reading the text.

\hypertarget{web-technologies}{%
\chapter{Web Technologies}\label{web-technologies}}

Prequisites for learning D3 include a basic understanding of:

\begin{enumerate}
\def\labelenumi{\arabic{enumi}.}
\item
  HTML -- the language of the web
\item
  CSS -- used for styling web pages, and more importantly for our
  purposes, selecting elements on a page or in a graphic
\item
  SVG -- the graphics format that we will be using
\item
  JavaScript -- language for making web pages interactive, animated,
  code is executed when page is opened or refreshed
\end{enumerate}

As our focus is D3, not building complex web sites with multiple pages,
we will learn minimal amounts of these technologies on a need to know
basis.

\hypertarget{html}{%
\section{HTML}\label{html}}

\hypertarget{learn-to-use-chrome-developer-tools}{%
\section{Learn to use Chrome Developer
Tools}\label{learn-to-use-chrome-developer-tools}}

\begin{enumerate}
\def\labelenumi{\arabic{enumi}.}
\item
  Opening Chrome: click \emph{View, Developer, JavaScript Console}.
  (There are
  \href{https://developers.google.com/web/tools/chrome-devtools/open}{other
  ways to open Chrome DevTools}).
\item
  Using the JavaScript Console to execute code, not necessarily
  connected with the current page.
\item
\end{enumerate}

\hypertarget{back-to-d3}{%
\chapter{Back to D3}\label{back-to-d3}}

Some \emph{significant} applications are demonstrated in this chapter.

\hypertarget{example-one}{%
\section{Example one}\label{example-one}}

\hypertarget{example-two}{%
\section{Example two}\label{example-two}}

\hypertarget{final-words}{%
\chapter{Final Words}\label{final-words}}

We have finished a nice book.

\hypertarget{scales}{%
\chapter{Scales}\label{scales}}


\end{document}
